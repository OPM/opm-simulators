\section{Specifying Parameters}
\label{sec:inputFiles}

As already mentioned in the previous section, \eWoms allows to specify
a number of parameters at run time. A description of these parameters
can usually be obtained by passing the \texttt{--help} parameter to the
executable. The values used for these parameters are printed to the
terminal during the start-up of the simulation, e.g. the if the \texttt{lens\_immiscible}
simulation is run with the parameters \texttt{--end-time=30e3 --foo-param=123}, it will print an output similar to
\begin{lstlisting}[style=Bash]
###########
# Parameters specified at run-time
###########
EndTime="30e3"
###########
# Parameters with use compile-time fallback values
###########
AmgCoarsenTarget="5000"
...
VtkWriteSaturations="1"
VtkWriteTemperature="1"
VtkWriteViscosities="0"
###########
# Parameters unknown to the simulation
###########
FooParam="123"
\end{lstlisting}

This can be directly copy-and-pasted into a file
(e.g. \texttt{myparams.ini}). The simulation can be instructed to use
these parameters by passing the \texttt{--parameter-file=myparams.ini}
parameter the next time it is run. Of course, this file can also be
adapted.

An additional hint: Always watch out for parameters which are unknown
to the simulation, as this is \emph{very} useful for spotting typos.

%%% Local Variables:
%%% mode: latex
%%% TeX-master: "ewoms-handbook"
%%% End:
