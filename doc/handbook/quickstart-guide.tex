\section[Quick start guide]{Compiling and running a sample application}
\label{quick-start-guide}

The previous section showed how to install and compile \eWoms. This
chapter gives you a very brief introduction how to run a first test
application and how to visualize the output files it produces. Only the
rough steps will be described here; More detailed explanations can be
found in the tutorials in the following chapter.

\begin{enumerate}
\item Go to the directory \texttt{test}. There, various test
  application folders can be found. Let us consider as example
  the one for the fully-implicit models, \texttt{boxmodels}:
\item Enter the folder \texttt{boxmodels}.
\item By default, the \texttt{dunecontrol} command only compiles the
  parts of \Dune modules that are necessary to build modules depending
  on the given module. For \eWoms, \texttt{dunecontrol} does not build
  anything by default, because \eWoms only provides \Cplusplus
  template classes but no libraries that need compilation. To compile
  the test simulation for the "lens" problem which uses the fully-implicit
  model that assumes immisciblility, enter
\begin{lstlisting}[style=Bash]
make lens_immiscible
\end{lstlisting}

You may also compile all available tests which use the box scheme by entering
\begin{lstlisting}[style=Bash]
make check
\end{lstlisting}

This takes quite some time, but if you have a multi-core processor
with \$N cores, you can considerably speed up compilation by using all
of available cores by using
\begin{lstlisting}[style=Bash]
make -j $N check
\end{lstlisting}
%$

\item If everything was compiled correctly, there should be an
  executable called '\texttt{lens{\_}immiscible}'. To run the simulation,
  simply execute it, i.e. enter
\begin{lstlisting}[style=Bash]
./lens_immiscible
\end{lstlisting}

You may also want to change some parameters from the command line. For
example, if you want to change the time up to which the problem is
simulated to $30.000$ seconds, use
\begin{lstlisting}[style=Bash]
./lens_immiscible --end-time=30e3
\end{lstlisting}

You can also get a list of parameters recognized by the
simulation together with a brief description, by running
\begin{lstlisting}[style=Bash]
./lens_immiscible --help
\end{lstlisting}

\item After this, the simulation should start and produce some output
  on the terminal. It is possible to interrupt it at any time by
  pressing \texttt{<Ctrl>+<C>}.

\item The actual output files produced by the simulation are a series
  of \texttt{.vtu} files and a \texttt{.pvd} file. Each \texttt{.vtu}
  file contains "visualization ready" data for a single time step,
  while the \texttt{.pvd} file "stitches" these files together into a
  coherent data set. For example, the \texttt{.pvd} holds the
  simulation time at which a given time step was produced, that can
  later be used for visualization.
\item You can now display the result of the simulation using the
  visualization tool ParaView (or, if you prefer, VisIt). Just type
  \texttt{paraview} in the console and open the \texttt{.pvd} file. On
  the left hand side, you should now be able to click the green
  ``Apply'' button. Once you have done this, the visualization of the
  simulation result appears on the screen and you can click the
  ``play'' button in the toolbar to view display its evolution over
  time.  Also note that you can choose the visualized quantity
  in the toolbar. For the lens problem, the most interesting quantities
  are probably the saturations.
\item Play a bit around to make your self familiar with the
  visualization tool of your choice as you will be using it a lot.
\end{enumerate}

%%% Local Variables:
%%% mode: latex
%%% TeX-master: "ewoms-handbook"
%%% End:
