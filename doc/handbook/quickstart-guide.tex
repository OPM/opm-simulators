\section[Quick start guide]{Quick start guide: The first run of a test application}\label{quick-start-guide}

PRELIMINARY DRAFT, tbc.: The previous chapter showed, how to install and compile \Dumux. This chapter shall give a very brief introduction, how to run a first test application, how to choose the materials, where to change boundary conditions, and how to visualize the first output files. Moreover, it will be explained, how to generate an own application folder. Only the rough steps will be described here. More detailed explanations can be found in the tutorials in the following chapter.

\begin{enumerate}
 \item Go to the directory \tt dune-mux/test\rm . There, various test application folders can be found. Let us consider the \tt test{\_}twophase \rm example.
 \item Enter the folder test{\_}twophase. In order to run the simulation, type\\ 
\tt ./test{\_}twophase grids/unitcube2.dgf 1000 10\\
\rm into the console. The parameters that are used in the command are the external grid (in this case a 2D rectangular grid), the end time of the simulation and the initial timestep. The parameters that have to be entered when calling the application can be specified in the application file (here: test{\_}twophase.cc).
 \item The simulation starts and produces some .vtu output files and also a .pvd file. The .pvd file can be used to examine time series and summarizes the .vtu-files.
 \item You can display the results using the previously installed visualization tool ParaView. Just type \tt paraview \rm in the console and open the .pvd file. On the left hand side, you can choose the desired parameter to be displayed.
\end{enumerate}
% 
%
%


