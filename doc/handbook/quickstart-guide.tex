\chapter[Quick start guide]{Quick start guide: The first run of a test application}

PRELIMINARY DRAFT, tbc.: In the previous chapter, it was explained how to install and compile \Dumux. This chapter shall give a very brief introduction, how to run a first test application and to visualize the first output files. The rough steps will be described here. More detailed explanations can be found in the tutorials.

\begin{enumerate}
 \item Go to the directory dune-mux/test. There, various test application folders can be found. Let us consider the test{\_}twophase example.
 \item Enter the folder test{\_}twophase and type for example 'test{\_}twophase grids/unitcube2.dgf 1000 10'. The parameters that occur here are the external grid (in this case a 2D square), the end time of the simulation and the initial timestep.
 \item Then, the simulation starts and produces some .vtu output files and also a .pvd file. This .pvd file can be used to examine time series and contains information about several .vtu-files.
 \item You can display the results using the previously installed visualization tool ParaView. Just type paraview into the console.
\end{enumerate}


