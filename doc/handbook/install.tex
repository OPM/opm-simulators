\chapter{Detailed Installation Instructions}
\label{install}

\section{Preliminary remarks}

In this section about the installation of \eWoms it is assumed that
you work on a Unix or Linux compatible operating system and that you
are familiar with the use of a command line shell. Installation means
that you unpack \Dune together with \eWoms in a certain directory.
Then, you compile it in that directory tree in which you do the
further work, too. You also should know how to install new software
packages or you should have a person on hand who assist you with
that. In section \ref{sec:prerequisites} we list some prerequisites
for running \Dune and \eWoms.  Please make sure to fulfill them before
you proceed. In addition, section \ref{sec:external-modules-libraries}
provides some details on optional libraries and modules.

In a technical sense \eWoms is a module of \Dune.  Thus, the
installation procedure of \eWoms is the same as that of \Dune (besides
using different locations to retieve the source code).  Details
regarding the installation of \Dune are provided on the \Dune
website~\cite{DUNE-INST}. If you are interested in more details about
the build system that is used, they can be found in the {\Dune}
build system howto \cite{DUNE-BS}.

All \Dune modules, including \eWoms, get extracted into a common
directory. In the following, we refer to that directory as {\Dune}
base directory or, in short, as {\Dune}-base.  If it is used as
directory path of a shell command it is typed as
\texttt{dune-base}.  For the actual location of the {\Dune} base
directory on your file system, you may chose any valid directory name for
which you have write permissions.

Source code files for each \Dune module are contained in their own
subdirectory within {\Dune}-base.  We name this directory of a certain
module \emph{module base directory} or \texttt{module-base}
if it is a directory path, e.\,g. for the module \texttt{ewoms} these
names are \emph{ewoms base directory} respective
\texttt{ewoms-base}.  The real directory names for the
modules can be chosen arbitrarily. In this manual they are the same as
the module name or the module name extended by a version number
suffix.  The name of each \Dune module is defined in the file
\texttt{dune.module}, which is in the base directory of the respective
module. This should not be changed by the user.  It is allowed to have
own files and directories in \Dune-base, which are not related to
\Dune's needs.

After installing source code for all relevant \Dune modules including
\eWoms, \Dune is being built by the shell-command \texttt{dunecontrol}
which is part of the \Dune build system. The \Dune build system is a
front-end of to the GNU build system adapted to the needs of \Dune.

\section{Prerequisites} \label{sec:prerequisites}

The GNU compiler collection with \Cplusplus support (\texttt{g++}) and
the tools of the GNU build system \cite{GNU-BS}, also known as GNU
autotools (\texttt{autoconf}, \texttt{automake}, \texttt{autogen},
\texttt{libtool}), as well as the GNU variant of \texttt{make} called
gmake must be available in a recent enough version. For example Ubuntu
Linux provides these tools by means of the packages \texttt{autoconf},
\texttt{automake}, \texttt{libtool} and the  package
\texttt{build-essential} includes the \Cplusplus compiler
\texttt{g++} and \texttt{make}.

At the time of writing this manual, the minumum version required for
\texttt{g++} is 4.4.0, \texttt{automake}  the documentation by setting the switch
\texttt{--disable-documentation} in the \texttt{CONFIGURE\_FLAGS} of
the building options (see Chapter \ref{buildIt}).  Additional parts of
documentation are contained within the source code files as specially
formatted comments.  Extracting them can be done using the tool
\texttt{doxygen} (version $\geqslant$ 1.8.2 works).  See for this
optional step section \ref{sec:build-doxy-doc}.

Depending on whether you are going to use external libraries and
modules for additional \Dune features, additional software packages
may be required. Some hints on that are given in Section
\ref{sec:external-modules-libraries}.

To access the git or subversion software repositories, git and
subversion clients are required. We recommend to use the git
command-line client \texttt{git} version 1.7.12 or newer with the
subversion integration plugin \texttt{git-svn} enabled~\cite{GIT-HP}.

\begin{table}
\centering
\caption{Ubuntu package names for Ubuntu 12.04}
\begin{tabular}{llll}
\toprule
\textbf{purpose} & \textbf{package names} \\
\midrule
general: & git & git-svn & libtool \\
& automake & build-essential & libboost-all-dev \\
& texlive-latex-base & doxygen & csh\\
& gfortran & \\
\midrule
for alberta: & freeglut3-dev &  \\
\midrule
for parallel use: & openmpi-common & mpi-default-bin & mpi-default-dev \\
\midrule
for parallel UG: & flex & bison & \\
\midrule
for parallel alberta: & libblas-dev &\\
\midrule
for debugging: & valgrind &\\
\bottomrule
\end{tabular}
\label{tbl:ubuntu-pkg}
\end{table}

\begin{table}
\centering
\caption{Package names for openSUSE 12.2. For this distribution, the {\em science} package repository needs to be added. The science repository is available from \url{http://download.opensuse.org/repositories/science/openSUSE_12.2/}. }
\begin{tabular}{llll}
\toprule
\textbf{purpose} & \textbf{package names} \\
\midrule
git & git-svn & dune-istl-devel & dune-grid-devel \\
dune-localfunctions-devel & \\
\bottomrule
\end{tabular}
\label{tbl:opensuse-pkg}
\end{table}

\section{Obtaining source code for \Dune and \eWoms}

As stated above, the \eWoms is based on the \Dune release 2.2,
comprising the core modules \texttt{dune-common},
\texttt{dune-geometry}, \texttt{dune-grid}, \texttt{dune-istl} and
\texttt{dune-localfunctions}.  For working with \eWoms, these modules
are required.

Two possibilities exist to get the source code of \Dune and \eWoms.
Firstly, released versions \Dune and \eWoms can be downloaded as tar
files from the \Dune and \eWoms websites. They have to be extracted as
described in the next paragraph.  Secondly, the most recent source
code can be obtained by directly downloading it from its respective
source-code repository. This method is described in the subsequent
section.

\paragraph{Obtaining the software by installing tar files}

The slightly old-fashionedly named tape-archive-file, shortly named
tar file or tarball, is a common file format for distributing
collections of files contained within these archives.  The extraction
from the tar files is done as follows: Download the tarballs from the
respective \Dune~\cite{DUNE-HP} (version 2.2.0) and
\eWoms~\cite{EWOMS-HP} websites.  Then, create the {\Dune} base
directory, named \texttt{~/src} in the example below. Then, extract the
content of the tar files, e.\,g. with the command-line program
\texttt{tar}.  This can be achieved by the following shell
commands. Replace \texttt{path\_to\_tarball} with the directory name
where the downloaded files are actually located.  After extraction,
the actual name of the \emph{ewoms base directory} is
\texttt{ewoms-2.2}.

\begin{lstlisting}[style=Bash]
$ mkdir ~/src
$ cd ~/src
$ tar xzvf path_to_tarball_of/dune-common-2.2.0.tar.gz
$ tar xzvf path_to_tarball_of/dune-grid-2.2.0.tar.gz
$ tar xzvf path_to_tarball_of/dune-geometry-2.2.0.tar.gz
$ tar xzvf path_to_tarball_of/dune-istl-2.2.0.tar.gz
$ tar xzvf path_to_tarball_of/dune-localfunctions-2.2.0.tar.gz
$ tar xzvf path_to_tarball_of/ewoms-2.2.0.tar.gz
\end{lstlisting}

\paragraph{Obtaining \Dune and \eWoms from software repositories}

Direct access to a software repositories for downloading code can be
convenient to always follow the changes of done to the software
project. Usually, source code repositories are structured in
branches. One of these branches is commonly used to include the latest
features of the software, and there is normally one branch for each
release which only receives fixes for bug which have been discovered
after the software was released. The following text describes how to
retrieve the source code of \Dune and \eWoms from such release
branches.

The \Dune project uses Apache Subversion~\cite{APACHE-SUBVERSION-HP}
to manage its software repositories, while \eWoms -- being part of the
open porous media project -- opted to use the git source code
management system~\cite{GIT-HP}. Fortunately, git ships with a
subversion plugin, so all modules can be managed using git.

In the technical language of Apache Subversion \emph{cloning a certain
  software repository} means nothing more then fetching a local copy
of the software repository and placing it on the local file system.
In addition to the software some more files for the use of the
software revision control system itself are created. They are kept in
a directory named \texttt{.git} at the base directory of the cloned
software repository.

The installation procedure is structured as follows: Create a {\Dune} base
directory (named \texttt{~/src} in the lines below).  Then, enter the
previously created directory and check out the desired modules.  As
you see below, the check-out uses two different servers for getting
the sources, one for \Dune and one for \eWoms.  The \Dune modules of
the stable 2.2 release branch are cloned similarly as described on the
\Dune website~\cite{DUNE-DOWNLOAD-SVN}:

\begin{lstlisting}[style=Bash]
mkdir ~/src
cd ~/src
for DUNE_MODULE in common geometry grid istl localfunctions; do
  git svn clone \
   https://svn.dune-project.org/svn/dune-$DUNE_MODULE/branches/release-2.2 \
   dune-$DUNE_MODULE
done
\end{lstlisting}

The newest and maybe unstable developments are also provided in these
repositories in a folder called \emph{trunk}. Please check the \Dune
website \cite{DUNE-DOWNLOAD-SVN} for further information. However, the
current \eWoms release is based on the stable 2.2 \Dune release and it
might misbehave using the the newest version of \Dune.

The \eWoms module is checked out as described below (see also the
\eWoms website \cite{EWOMS-HP}).  Its source tree has to be created in
the \Dune-base directory, where the \Dune modules have also been
cloned into. Subsequently, the next command is executed there,
too. The directory which holds the \eWoms module is called
\texttt{ewoms} here.

\begin{lstlisting}[style=Bash]
cd ~/src
git clone git://github.com/OPM/ewoms.git ewoms
\end{lstlisting}

\section{Building the doxygen documentation}
\label{sec:build-doxy-doc}

Doxygen documentation is done by especially formatted comments
integrated in the source code, which can get extracted by the program
\texttt{doxygen}. Beside extracting these comments, \texttt{doxygen}
builds up a web-browsable code structure documentation like class
hierarchy of code displayed as graphs, see \cite{DOXYGEN-HP}.

Building the doxygen documentation of a module is done as follows,
provided the program \texttt{doxygen} is installed: Set in building
options the \texttt{--enable-doxygen} switch.  This is either
accomplished by adding it in \texttt{dunecontrol} options-file to
\texttt{CONFIGURE\_FLAGS}, or by adding it to \texttt{dunecontrol}'s
command-line-argument \texttt{--configure-opts}.  After running
\texttt{dunecontrol} enter in module's base directory the subdirectory
\texttt{doc/doxygen}.  You then run the command \texttt{doxygen}
within that directory. Point your web browser to the file
\texttt{module-base-directory/doc/doxygen/html/index.html} to read the
generated documentation.  For all \Dune-modules that described here,
the doxygen documentation can be generated analogously.

\begin{lstlisting}[style=Bash]
cd ~/src/ewoms/doc/doxygen
doxygen
firefox html/index.html
\end{lstlisting}

\section{Building the \eWoms handbook}

If the \texttt{--enable-documentation} switch has been set in the configure flags of
\texttt{dunecontrol}, watch for a summary line of the build script that reads
\begin{lstlisting}[style=Bash]
Build eWoms handbook....: yes
\end{lstlisting}

If this is the case, the handbook should be automatically build by
\texttt{dunecontrol} and can be found at
\texttt{\$EWOMS\_BASE/doc/handbook/ewoms-handbook.pdf}.

If the summary line says that the handbook is not going to be build,
then you usually have to install additional \LaTeX packages.

\section{External libraries and modules}
\label{sec:external-modules-libraries}

The libraries described below provide additional functionality but are
not generally required to use \eWoms.  If you are going to use an
external library, also check the information provided on the \Dune
website~\cite{DUNE-EXT-LIB} for additional hints.  If you are going to
use an external \Dune module, the website listing external
modules~\cite{DUNE-EXT-MOD} can also be helpful.

Some external libraries have additional dependencies which can also be
used by \Dune.  Also, some libraries, such as BLAS or MPI might have
multiple versions installed on the system.  To avoid problems, you
should make sure that all external libraries use the same dependencies
as \Dune.

In the following list, you can find some external modules and external
libraries, and some more libraries and tools which are prerequisites
for their use.

\begin{itemize}
\item \textbf{ALBERTA}: External library which can be used as an
  additional grid manager. ALBERTA stands for ``\textbf{A}daptive multi \textbf{L}evel finite element toolbox
  using \textbf{B}isectioning refinement and \textbf{E}rror control by \textbf{R}esidual
  \textbf{T}echniques for scientific \textbf{A}pplications''. Building it requires a
  Fortran compiler, for example \texttt{gfortran}. It can be downloaded at:
  \texttt{\url{http://www.alberta-fem.de}}.

\item \textbf{ALUGrid}: External library for use as grid. ALUGrid
  requires by a \Cplusplus compiler like \texttt{g++} to be build. If
  you want to build a parallel version, you will need an MPI library
  like, for example \texttt{openmpi}, installed on your system. The
  parallel version needs also a graph partitioner, such as
  \texttt{METIS}.

  Download:
  \texttt{\url{http://aam.mathematik.uni-freiburg.de/IAM/Research/alugrid}}

\item \textbf{SuperLU}: External library for solving linear
  equations. SuperLU is a general purpose library for the direct
  solution of large, sparse, non-symmetric systems of linear
  equations.

  Download:
  \texttt{\url{http://crd.lbl.gov/~xiaoye/SuperLU}}.

\item \textbf{UG}: External library which can be used as an additional
  grid manager. UG is a toolbox for \textbf{U}nstructured
  \textbf{G}rids: For \eWoms it has to be build by GNU buildsystem and
  a \Cplusplus compiler. That's why \Dune specific patches need
  applied before use. Building it makes use of the tools
  \texttt{lex}/\texttt{yacc} or the GNU variants
  \texttt{flex}/\texttt{bison}.

  Website:
  \texttt{\url{http://atlas.gcsc.uni-frankfurt.de/~ug/}}\\
  Further information:
  \texttt{\url{http://www.dune-project.org/external_libraries/install_ug.html}}\\

\end{itemize}

The following are dependencies of some of the external libraries. You
will need them depending on which modules of \Dune and which external
libraries you use.

\begin{itemize}
\item \textbf{MPI}: The parallel version of \Dune and also some of the
  external dependencies need MPI when they are going to be built for
  parallel computing. \texttt{Openmpi} version $\geqslant$ 1.4.2 and
  \texttt{MPICH} in a recent version have been reported to work.

\item \textbf{lex/yacc} or \textbf{flex/bison}: These are quite common
  developing tools, code generators for lexical analyzers and
  parsers. This is a prerequisite for compiling UG.

\item \textbf{BLAS}: Alberta makes use of BLAS (acronym for
  ``\textbf{B}asic \textbf{L}inear \textbf{A}lgebra
  \textbf{S}ubprograms). Thus install GotoBLAS2, ATLAS,
  non-optimized BLAS or a BLAS library provided by a computer
  vendor. Take care that the installation scripts select the intended
  version of BLAS. For further information, see
  \texttt{\url{http://en.wikipedia.org/wiki/Basic_Linear_Algebra_Subprograms}}.

\item \textbf{GotoBLAS2}: This is an optimized BLAS library. It does
  not provide optimized routines for all modern processors, but quite
  a broad range. Also, its license is now open. A Fortran compiler
  like \texttt{gfortran} is needed to compile it.

  Available at
  \texttt{\url{http://www.tacc.utexas.edu/tacc-projects/gotoblas2/}}.

\item \textbf{METIS}: This is a dependency of ALUGrid, if you are
  going to run it parallel.

  Available for non-commercial use at
  \texttt{\url{http://glaros.dtc.umn.edu/gkhome/metis/metis/overview}}

\item \textbf{Compilers}: Besides \texttt{g++} \Dune and \eWoms can
  also be built with the \texttt{clang++} compiler originating from
  the LLVM project. If you try other compilers, be aware that in addition to a
  \Cplusplus compiler, C and Fortran compilers are needed to compile
  some external libraries.
\end{itemize}

%%% Local Variables:
%%% mode: latex
%%% TeX-master: "ewoms-handbook"
%%% End:
