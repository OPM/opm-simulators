\section{Installation} 
\label{install}

For the installation of DuMu$^\text{x}$, the following steps have to be performed:  

\paragraph{Checkout of the core modules}
From version 2.0 of DUNE, it was decided to stick to stable DUNE releases, comprising the core modules 
\texttt{dune-common}, \texttt{dune-grid}, \texttt{dune-istl}, \texttt{dune-localfunctions}, and the external \texttt{dune-pdelab}.  
First, create a directory where all the DUNE and \Dumux modules will be stored in. Then, enter the previously created folder. 
The checkout has to be performed as described on 
the DUNE webpage, \cite{DUNE-HP}: 
\begin{itemize}
\item \texttt{svn checkout https://svn.dune-project.org/svn/dune-common/releases/2.0 dune-common}
\item \texttt{svn checkout https://svn.dune-project.org/svn/dune-grid/releases/2.0 dune-grid}
\item \texttt{svn checkout https://svn.dune-project.org/svn/dune-istl/releases/2.0 dune-istl}
\item \texttt{svn checkout https://svn.dune-project.org/svn/dune-localfunctions/releases/2.0 dune-localfunctions}
\item \texttt{svn checkout https://svn.dune-project.org/svn/dune-pdelab/branches/2.0snapshot dune-pdelab}
\end{itemize} 
%The \Dumux webpage \cite{dumux-hp} contains the revision numbers of the core modules 
%for which compatibility to \Dumux can be guaranteed. 

Additionally, it is highly recommended that you also checkout the \texttt{dune-grid} HOWTO 
by 
\begin{center}
\texttt{svn checkout https://svn.dune-project.org/svn/dune-grid-howto/releases/2.0 dune-grid-howto}
\end{center}
and work yourself through that tutorial in order to get an understanding of the Dune grid interface. 

\paragraph{Obtain \Dumux}
Two possibilities exist to obtain \Dumux. Either you register on the \Dumux website \cite{dumux-hp}, and download the tarball of a stable release, or 
you checkout the (unstable) version from the \Dumux repository.

\paragraph{Download of the \Dumux tarball from the website}
The \Dumux tarball can be extracted as usual in the same folder, where you have performed the checkout of the DUNE modules. You can than skip the following paragraph.

\paragraph{Checkout of \Dumux from the SVN repository} 
On the repository, the \Dumux project is divided into a stable and a developers part which should stay seperate
on the file system. You can checkout both parts of \Dumux via 
\begin{itemize}
 \item \texttt{svn checkout --username=anonymous --password='' \\ 
      \hspace{2cm} svn://svn.iws.uni-stuttgart.de/DUMUX/dumux/trunk dumux}
 \item \texttt{svn checkout --username=anonymous --password='' \\
      \hspace{2cm} svn://svn.iws.uni-stuttgart.de/DUMUX/dune-mux/trunk dumux-devel}
%\item \texttt{svn checkout svn+ssh://luftig/home/svn/DUMUX/external/trunk external}
\end{itemize} 
The repositories allow a read-only access. If you also want to commit new developments to the repositories, you can ask one of the LH$^2$ administrators to add you to the svn group ( \texttt{svndune}).

\paragraph{Required external modules}
Among others, the following modules provide additional functionality, but are not required to run \Dumux:
\begin{itemize}
 \item grids: Alberta, AluGrid, UG
 \item solvers: Pardiso, SuperLU
 \item BLAS and METIS library 
\end{itemize}
If you are working on a LH$^2$ computer, you can checkout a set of external modules via \\
\texttt{svn checkout svn+ssh://luftig/home/svn/DUMUX/external/trunk external}. \\
To install them all, execute the install script in the folder \texttt{external}:
\begin{center}
\texttt{./installExternal.sh all}
\end{center}
If you like to only install some of the external software, you can choose one of the 
corresponding options instead of \texttt{all}: \texttt{alberta, alu, blas, metis, ug}.
Please also refer to the DUNE webpage for additional details, \cite{DUNE-HP}. 


\paragraph{Installation on a non-IWS PC}
If \Dumux is not to be used on a PC maintained by LH$^2$ it is probable that one needs to install additional software. 
Here is a (hopefully complete) list of packages that needed to  be installed on a fresh ubuntu 10.04 in order to get \verb+dunecontrol+ running.
\begin{itemize}
  \item build-essential
  \item openmpi-common / openmpi-bin ...
  \item gfortran
  \item libtool
  \item libboost-all-dev
  \item bison
  \item flex
  \item gcc-4.3 (hard coded into installExternal.sh)
  \item blas ???
  \item pardiso
\end{itemize}


\paragraph{Build the external modules} 
If you obtained a \Dumux tarball, you can skip this part. 

\begin{turn}{45}\textbf{Attenschion}\end{turn}
\begin{turn}{45}\textbf{Attenschion}\end{turn}
\begin{turn}{45}\textbf{Attenschion}\end{turn}
\begin{turn}{45}\textbf{Attenschion}\end{turn}
\begin{turn}{45}\textbf{Attenschion}\end{turn}

{\it

  This  part needs to be extended in order to show which package is necessary for what purpose, 
  so the user can decide whether getting all the grmblDamnItDamnbndfjk 
  moduls working is worth it! \\

  So could somebody please write something about the respective external modules!\\ 

Also, the gcc version 4.3 is hard coded into the skript. }

\begin{turn}{45}\textbf{Attenschion}\end{turn}
\begin{turn}{45}\textbf{Attenschion}\end{turn}
\begin{turn}{45}\textbf{Attenschion}\end{turn}
\begin{turn}{45}\textbf{Attenschion}\end{turn}
\begin{turn}{45}\textbf{Attenschion}\end{turn}

The external modules have to be built first. They consist of Alberta, ALUGrid, UG, and METIS.

\paragraph{Build \Dune and \Dumux}
\label{buildIt}
If you want to compile in the debugging mode, type in the folder \texttt{DUMUX}: 
\begin{center}
\texttt{./dune-common/bin/dunecontrol --opts=\$DUMUX\_ROOT/debug.opts all}
\end{center}

Otherwise you can type
\begin{center}
\texttt{./dune-common/bin/dunecontrol --opts=\$DUMUX\_ROOT/optim.opts all}
\end{center}
in order to get an optimized compilation (better performance, but no possibility to use a debugger).

This uses the DUNE buildsystem. If it does not work, please have a look at
the file \texttt{INSTALL} in the \Dumux root directory (if you use
SVN, this \texttt{\$DUMUX\_ROOT} is usually \texttt{dumux}, if you use
a released version it is usually \texttt{dumux-VERSION}). You can also
find more information in the DUNE Buildsystem HOWTO located at the
DUNE webpage, \cite{DUNE-HP}.  Alternatively, the tool CMake can be
used to build \Dumux. Please check the file \texttt{INSTALL.cmake} for
details.
