In the unsaturated zone Richards` equation can be used. Gas has resistance against the water flow in porous media. However, viscosity of air is about 1\% of the viscosity of water, which makes it highly mobile compared to the water phase. Therefore, in Richards` equation only water phase with capillary effects are considered, where pressure of the gas phase is set to a reference pressure (${p_n}_{ref}$).

\begin{align*} \varrho \hspace{1mm} \phi \hspace{1mm} \frac{\partial S_w}{\partial p_c} \frac{\partial p_c}{\partial t} - \nabla \cdot (\frac{kr_w}{\mu_w} \hspace{1mm} \varrho_w \hspace{1mm} K \hspace{1mm} (\nabla p_w - \varrho_w \hspace{1mm} \vec{g})) \hspace{1mm} = \hspace{1mm} q \\ ,where \hspace{1mm} p_w = {p_n}_{ref} - p_c \end{align*} Here $ p_w $, $ p_c $, $ p_n $ denotes water pressure, capillary pressure, non-wetting phase reference pressure repectively.

To overcome convergence problem $ \frac{\partial S_w}{\partial p_c} $ is taken from old iteration step. 
